\documentclass[UTF8]{ctexart}  %使用中文版的article文档类型排版,并选择UTF8编码格式
\usepackage{amsmath}  %使用宏包,这里使用的是调用公式宏包,可以调用多个宏包
\usepackage{graphicx}   %我们先添加宏包插图,方便接下来添加图片
\usepackage{amsmath}  %使用宏包,这里使用的是调用公式宏包,可以调用多个宏包
\newtheorem{thm}{定理}  %定义标题为定理的定理类环境thm
\newcommand\degree{^\circ}    %定义新命令degree

\begin{document}  %开始写文章
\title{杂谈勾股定理}  %大括号里填写标题
\author{张三}  %大括号里填写作者姓名
\date{\today}    %大括号里填写\today会自动生成当前的日期
\maketitle     %我们写了以上内容以后一定要添加这个,制作标题,否则上面的内容都是无效的。

\begin{abstract}         %摘要部分
\small{这是一篇关于勾股定理的小论文}
\end{abstract}

\tableofcontents  %表示目录部分开始
\section{勾股定理在古代}  %目录的前缀页面都会自动排版不需要手动排版
\section{勾股定理的近代形式}
\addcontentsline{toc}{section}{参考文献}  %用来添加文献的标准方式
\section{附录}  %要写的附录


\label{sec:ancient}  %这是添加书签的命令,一会儿要用到
\small{西方称勾股定理为毕达哥拉斯定理,将勾股定理的发现归功于公元前6世纪的毕达哥拉斯学派}\cite{Kline}。该学派得到了一个法则,可以求出可排成直角三角形三边的三元数组。毕达哥拉斯学派没有书面著作,该定理的严格表述和证明则见于欧几里得\footnote{欧几里得,公元前 330——275 年。}《几何原本》的命题47:“直角三角形斜边上的正方形等于两直角边上的两个正方形之和。”证明是用面积做的。\%这里cite{Kline}是为了一会添加引用文献标记用的,\footnote{是在文章下面自动添加注释的命令,换行可以使用空一行的方法。}

\small 我国《周髀算经》载商高(约公元前12世纪)答周公问:

\footnotesize\centering{勾广三,股修四,径隅五}。%这里调整字体并使字体居中

\small 又载陈子(公元前 7——6 世纪)答荣方问:


\footnotesize\centering{若求邪至日者,以日下为勾,日高为股,勾股各自乘,并开方而除之,得邪至日。}

\small 较古希腊更早。后者已经明确道出勾股定理的一般形式。图\ref{fig:xiantu}是我国古代对勾股定理的一种证明\cite{quanjing}   %\ref{fig:xiantu}用于读取标签xiantu,cite用于引用参考文献quanjing
\begin{figure}[!ht]\centering   %添加图片环境的配置
\includegraphics[scale=0.5]{xiantu.jpg}    %添加图片,图片文件名为xiantu.jpg
\caption{宋赵爽在《周髀算经》注中作的弦图(仿制),该图给出了勾股定理的一个极具对称美的证明。\label{fig:xiantu}}  %在图片下面的文字说明
\end{figure}



\begin{thm}[\small{勾股定理}]  %开始定理环境
\small{直角三角形斜边的平方等于两腰的平方和。}

\small {可以用符号语言表述为:设直角三角形$ABC$,其中$\angle C=90\degree$,则有}   %$$之间为数学表达式的书写地方,我们定义的degree是为了写度数符号的。
\begin{equation}\label{eq:gougu}   %开始单行公式环境equation,并添加了书签gougu
\small AB^2=BC^2+AC^2.
\end{equation}
\end{thm}
\small 满足式 \eqref{eq:gougu} 的整数称为\emph{勾股数}。第 \ref{sec:ancient} 节所说毕达哥拉斯学派得到的三元数就是勾股数。下表给出一些较小的勾股数:%\eqref{eq:gougu}读入书签gougu;emph强调勾股数,ref读入书签ancient

\vspace{3mm}    %空一行
\begin{tabular}{|c|c|c|}\hline   %开始表格环境,{|c|c|c|}表示文字居中的三列,\hline...\hline表述画两条并排的水平线。\hline必须用于首行之前或者换行命令之后。
\small{}     %&是数据分割符号
3&4&5\\\hline
5&12&13\\\hline
\end{tabular}
\small($a^2+b^2=c^2$)

\begin{thebibliography}{99}    %参考文献开始
\bibitem{1}失野健太郎.几何的有名定理.上海科学技术出版社,1986.                    %参考文献1
\bibitem{quanjing}曲安金.商高、赵爽与刘辉关于勾股定理的证明.数学传播,20(3),1998.                  %参考文献2
\bibitem{Kline}克莱因.古今数学思想.上海科学技术出版社,2002.
\end{thebibliography}
\addcontentsline{toc}{section}{参考文献}
\begin{appendix}               %附录开始
\section{附录}
\small {勾股定理又叫商高定理,国外也称百牛定理。}
\end{appendix}



\end{document}  %结束写文章
